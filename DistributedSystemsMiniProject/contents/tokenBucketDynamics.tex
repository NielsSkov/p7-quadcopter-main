\section{Token Bucket Dynamics}
The dynamics of the filter can be described using the matrix $H$ as
\begin{flalign}
    P_n=P_{n-1}H
\end{flalign} 
%
where $P_n$ contains the probabilities of being in each of the states of the queue at each time in the queue and $P_{n-1}$ contains the probabilities in the previous time.

As time approaches infinity, $P_n$ converges. This converged vector of probabilities is called $\Pi$ and fulfills the balance equation 
\begin{flalign}
    \Pi=\Pi H \label{eq:pieq}
\end{flalign} 

To be able to calculate the mean queue length, mean waiting time and the packet loss probability, it is necessary to find $\Pi$, that gives the converged probabilities of each of the queue.
    
This is done solving \autoref{eq:pieq} and forcing $\sum_{i=0}^{L} \Pi$ (L is the queue length) to be 1.

The first step is to calculate H. It contains the probabilities of moving among the different states in the queue. Since the queue lengths are sampled each token period, $\frac{1}{T_r}$, the probabilities for a number of packets arriving can be calculated as
\begin{flalign}
    A_a=P(a_n=j)=\frac{(\lambda \frac{1}{T_r})^j}{\lambda !} \exp(-\lambda \frac{1}{T_r})
\end{flalign}
%
where $a_n$ is the number of arrivals from a Poisson stream in the service
period, $\lambda$ is the arrival rate of packets to the filter.

\begin{flalign}
H=
\begin{bmatrix}
A_0 & A_1 & A_2 & A_3 & ... & ... & 1-\sum_{0}^{L-1}A_j  \\
A_0 & A_1 & A_2 & A_3 & ... & ... & 1-\sum_{0}^{L-1}A_j \\
0   & A_0 & A_1 & A_2 & ... & ... & 1-\sum_{0}^{L-2}A_j \\
0   & 0   & A_0 & A_1 & ... & ... & 1-\sum_{0}^{L-3}A_j \\
... & ... & ... & ... & ... & ... & ... \\
0   & 0   & 0   & 0   & ... & A_0 & 1-A_0
\end{bmatrix}
\end{flalign}

Once it is calculated it is possible to solve for $\Pi$ in \autoref{eq:pieq}

The mean queue length, mean waiting time and packet loss probability are given by:
\begin{flalign}
    \bar{Q}=\sum_{i=0}^{L} \Pi_i i \\
    \bar{W}=\frac{\bar{Q}}{\lambda} \\
    PLP=\Pi_L
\end{flalign}

\subsection{Rear Camera}
The $H$ matrix in the case of the rear camera is $9\time9$ and can be seen in \autoref{}
\begin{flalign}
H_{RC}=
  \begin{bmatrix}
    A_0 & A_1 & A_2 & A_3 & A_4 & A_5 & A_6 & A_7 & 1-\sum_{0}^{7}A_j  \\
    A_0 & A_1 & A_2 & A_3 & A_4 & A_5 & A_6 & A_7 & 1-\sum_{0}^{7}A_j \\
    0   & A_0 & A_1 & A_2 & A_3 & A_4 & A_5 & A_6 & 1-\sum_{0}^{6}A_j \\
    0   & 0   & A_0 & A_1 & A_2 & A_3 & A_4 & A_5 & 1-\sum_{0}^{5}A_j \\
    0   & 0   & 0   & A_0 & A_1 & A_2 & A_3 & A_4 & 1-\sum_{0}^{4}A_j \\
    0   & 0   & 0   & 0   & A_0 & A_1 & A_2 & A_3 & 1-\sum_{0}^{3}A_j \\
    0   & 0   & 0   & 0   & 0   & A_0 & A_1 & A_2 & 1-\sum_{0}^{2}A_j \\
    0   & 0   & 0   & 0   & 0   & 0   & A_0 & A_1 & 1-\sum_{0}^{1}A_j \\
    0   & 0   & 0   & 0   & 0   & 0   & 0   & A_0 & 1-A_0
  \end{bmatrix}
\end{flalign}


\subsection{Multimedia System}
\begin{flalign}
H_{multimedia}=
\begin{bmatrix}
A_0 & A_1 & A_2 & A_3 & A_4 & 1-\sum_{0}^{4}A_j  \\
A_0 & A_1 & A_2 & A_3 & A_4 & 1-\sum_{0}^{4}A_j \\
0   & A_0 & A_1 & A_2 & A_3 & 1-\sum_{0}^{3}A_j \\
0   & 0   & A_0 & A_1 & A_2 & 1-\sum_{0}^{2}A_j \\
0   & 0   & 0   & A_0 & A_1 & 1-\sum_{0}^{1}A_j \\
0   & 0   & 0   & 0   & A_0 & 1-A_0
\end{bmatrix}
\end{flalign}