\chapter{Equations}

% Equation in one row without units

\begin{flalign}
	\eq{P(\vec{\theta})} {\frac{1}{2N}\sum_{k = 1}^{N} \left(y_{k} - y_{m_k}(\vec{\theta})\right)^2 } &
	\label{costFunctionEquation}
\end{flalign}

% Equation in one row with units. It also includes the description of the variables

\begin{flalign}
	\eq{J_F \vec{\ddot{\theta}_F}} { -B_F \vec{\dot{\theta}_F} + \vec{l_F} \times (m_F\cdot \vec{g}) + \vec{l_w} \times \vec{F} - \vec{\tau_m} + B_w \vec{\dot{\theta}_w}} \unit{N\cdot m}
	\label{frameModelEq}
\end{flalign}

\hspace{6mm} Where:\\
\begin{tabular}{ p{1cm} l l l}
	& \si{J_F} 					    	   & is the inertia of the frame                          &\unitWh{kg \cdot m^2} \\
	& \si{\vec{\ddot{\theta}_F}} & is the angular acceleration of the frame             &\unitWh{rad \cdot s^{-2}} \\
	& \si{B_F} 	                 & is the friction coefficient of the frame             &\unitWh{N \cdot m \cdot s \cdot rad^{-1}} \\
	& \si{\vec{\dot{\theta}_F}}  & is the angular velocity of the frame                 &\unitWh{rad \cdot s^{-1}} \\
	& \si{\vec{l_F}}             & is the length to center of mass of the frame         &\unitWh{m} \\
	& \si{m_F}                   & is the mass of the frame                             &\unitWh{kg} \\
	& \si{\vec{g}}							 & is the gravitational acceleration                    &\unitWh{m\cdot s^{-2}} \\
	& \si{\vec{l_w}}             & is the length to center off mass of the wheel        &\unitWh{m} \\
	& \si{\vec{F}}				  	   & is the force delivered to the frame from the wheel   &\unitWh{N} \\
	& \si{\vec{\tau_m}} 	       & is the torque delivered by the motor        &\unitWh{N \cdot m} \\
	& \si{B_w} 	                 & is the friction coefficient of the wheel             &\unitWh{N \cdot m \cdot s \cdot rad^{-1}} \\
	& \si{\vec{\dot{\theta}_w}}  & is the angular velocity of the wheel with respect to the frame                 &\unitWh{rad \cdot s^{-1}} \\
\end{tabular}

% More than one equation (one on each row) that share the reference number

\begin{flalign}
	\eq{ -F_x -m_w \cdot g }{ m_w \cdot \ddot{x} } &\nonumber\\
	\eq{ F_x }{ - m_w \cdot \ddot{x} - m_w \cdot g} &\nonumber\\
	\eq{ F_x }{ m_w \cdot (\  l_w \cdot cos( \theta_F )\ {\dot{\theta}_F}^{\ \ 2} + l_w \cdot sin( \theta_F )\ \ddot{\theta}_F \ ) - m_w \cdot g} \unit{N}
	\label{Fx}
\end{flalign}

% Equation in two rows

\begin{flalign}
	\eqOne{\ddot{\theta}_w} {\frac{(J_w+J_F+{l_w}^{2} \cdot m_w) \cdot (\tau_m - B_w \cdot \dot{\theta}_w)}{J_w \cdot (J_F+m_w \cdot {l_w}^{2})}}
	\eqTwo{- \frac{(m_F \cdot l_F + m_w \cdot l_w) \cdot g \cdot sin(\theta_F) - B_F \cdot \dot{\theta}_F}{J_F+m_w \cdot {l_w}^{2}}} \unit{rad \cdot s^{-1}}
	\label{WheelRotEq4}
\end{flalign}

% Examples of equations with matrices. \begin{vmatrix} represents the determinant and 
% \begin{bmatrix} is the normal one between brackets.

\begin{flalign}
	\si{\vec{l_w} \times \vec{F}} &=
	\begin{vmatrix}
		\ \si{\vec{\hat{i}}}                & \si{\vec{\hat{j}}}               & \si{\vec{\hat{k}}} \ \ \ \\ 
		\ \si{ l_w \cdot cos( \theta_F ) }  & \si{ l_w \cdot sin( \theta_F ) } & 0                  \ \ \ \\ 
		\ \si{ F_x }                        & \si{ F_y }                      & 0                  \ \ \  
	\end{vmatrix} \unit{N\cdot m}\\ \nonumber \\
	\si{ \vec{l_w} \times \vec{F} } &= 
	\begin{bmatrix}
		\ \si{ ( l_w \cdot sin( \theta_F) \cdot 0 - 0 \cdot F_y ) } \ \ \ \\
		\ \si{ ( l_w \cdot cos( \theta_F )\cdot 0 + 0 \cdot F_x  ) } \ \ \ \\
		\ \si{ ( l_w \cdot cos( \theta_F )\cdot F_y - l_w \cdot sin( \theta_F )\cdot F_x ) }
	\end{bmatrix} \unit{N\cdot m} \\ \nonumber\\
	\label{vectorDecomposition3}
\end{flalign}

% Three equation with matrices in a row

\begin{minipage}{0.32\linewidth}
	\begin{flalign}
		\vec{x} = 
		\begin{bmatrix}
			\theta_F \\
			\dot{\theta}_F \\ 
			\dot{\theta}_w \\
		\end{bmatrix}	\nonumber
		\label{xVector}
	\end{flalign}  
\end{minipage}\hfill
%\hspace{0.03\linewidth}
\begin{minipage}{0.32\linewidth}
	\begin{flalign}
		\vec{y} = 
		\begin{bmatrix}
			\theta_F \\
			\dot{\theta}_w \\
		\end{bmatrix}	\nonumber
		\label{yVector}
	\end{flalign}
\end{minipage}\hfill
%\hspace{0.03\linewidth}
\begin{minipage}{0.32\linewidth}
	\begin{flalign}
		\vec{u}= 
		\begin{bmatrix}
			\tau_m\\
		\end{bmatrix}
		\label{uVector}
	\end{flalign}
\end{minipage}\hfill



