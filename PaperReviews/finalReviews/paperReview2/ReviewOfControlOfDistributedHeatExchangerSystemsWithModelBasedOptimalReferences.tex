%implementing document formatting:
\documentclass[12pt,twoside,a4paper]{report}

% Select encoding of your inputs
\usepackage[utf8]{inputenc}

% Make latex understand and use the typographic
% rules of the language used in the document.
\usepackage[english]{babel}

% Use the vector font Latin Modern which is going
% to be the default font in latex in the future.
\usepackage{lmodern}

% Choose the font encoding
\usepackage[T1]{fontenc}

% Use color in tables
\usepackage[table]{xcolor}
\usepackage{array}
\usepackage{multirow}

% Load a colour package
\usepackage{xcolor}
\definecolor{aaublue}{RGB}{33,26,82}  %<--define aaublue
\definecolor{white}{RGB}{255,255,255} %<--define white

% The standard graphics inclusion package
\usepackage{graphicx}

\makeatletter
  \g@addto@macro\@floatboxreset\centering %<--centering all figures
\makeatother

\usepackage{adjustbox}

% Set up how figure and table captions are displayed
\usepackage{float}
\usepackage{caption}
\usepackage{subcaption}
\captionsetup
{
  justification = justified,         %<--centering caption with multiple lines
  font          = footnotesize, %<--set font size to footnotesize
  labelfont     = bf            %<--bold label (e.g., Figure 3.2) font
}
\captionsetup[subfigure]
{
  justification = centering, %<--centering subfigure caption text
  singlelinecheck=false,
  font = footnotesize        %<--font size for subfigures
} 

% Enable row combination in tables
\usepackage{multirow}

% Make space between table lines and text
\renewcommand{\arraystretch}{1.5}

% Enable commands like \st (strike out) and \hl (high light)
\usepackage{soul}

% Make the standard latex tables look so much better
\usepackage{array,booktabs}

% Enable the use of frames around, e.g., theorems
% The framed package is used in the example environment
\usepackage{framed}
\usepackage{colortbl}
\usepackage{longtable}
\usepackage{xcolor}
\usepackage{textcomp}

%-------MATHEMATICS---------------------------------
% Defines new environments such as equation,
% align and split 
\usepackage{amsmath}
\usepackage{relsize}
% Adds new math symbols
\usepackage{amssymb}
% Use theorems in your document
% The ntheorem package is also used for the example environment
% When using thmmarks, amsmath must be an option as well. Otherwise \eqref doesn't work anymore.
\usepackage[framed,amsmath,thmmarks]{ntheorem}
\usepackage{xifthen}%<--enables ifthenelse which is used in macros

\usepackage{siunitx} 
\sisetup{decimalsymbol=period}%<--\num{} will swich commas with periods
\sisetup{detect-weight}
%---------------------------------------------------

%-------PAGE LAYOUT---------------------------------
% Change margins, papersize, etc of the document
\usepackage[
  left=25mm,% left margin on an odd page %tidligere 25mm for baade right og left
  right=25mm,% right margin on an odd page
  top=35mm,
  ]{geometry}
  
% Modify how \chapter, \section, etc. look
% The titlesec package is very configureable
\usepackage{titlesec}
\makeatletter
\def\ttl@mkchap@i#1#2#3#4#5#6#7{%
    \ttl@assign\@tempskipa#3\relax\beforetitleunit
    \vspace{\@tempskipa}%<<<<<< REMOVE THE * AFTER \vspace
    \global\@afterindenttrue
    \ifcase#5 \global\@afterindentfalse\fi
    \ttl@assign\@tempskipb#4\relax\aftertitleunit
    \ttl@topmode{\@tempskipb}{%
        \ttl@select{#6}{#1}{#2}{#7}}%
    \ttl@finmarks  % Outside the box!
    \@ifundefined{ttlp@#6}{}{\ttlp@write{#6}}}
\makeatother

\titlespacing{\chapter}{0pt}{0pt}{10pt}
\titlespacing{\section}{0pt}{0pt}{-5pt}
\titlespacing{\subsection}{0pt}{8pt}{-5pt}
\titlespacing{\subsubsection}{0pt}{6pt}{-10pt}

\titleformat*{\section}{\normalfont\Large\bfseries\color{aaublue}}
\titleformat*{\subsection}{\normalfont\large\bfseries\color{aaublue}}
\titleformat*{\subsubsection}{\normalfont\normalsize\bfseries\color{aaublue}}

\usepackage{titlesec, blindtext, color}
%\color{gray75}{gray}{0.75}
\newcommand{\hsp}{\hspace{20pt}}
\titleformat{\chapter}[hang]{\Huge\bfseries}{\thechapter\hsp\textcolor{aaublue}{|}\hsp}{0pt}{\Huge\bfseries}

% Change the headers and footers
\usepackage{fancyhdr}
\setlength{\headheight}{15pt}
\pagestyle{fancy}
\fancyhf{} %delete everything
\renewcommand{\headrulewidth}{0pt} %remove the horizontal line in the header
%\fancyhead[RO,LE]{\color{aaublue}\small\nouppercase\leftmark} %even page - chapter title
\fancyhead[LO]{}
\fancyhead[RE]{} 
\fancyhead[CE]{}
\fancyhead[CO]{}
\fancyfoot[RE,LO]{\thepage}
%\fancyfoot[LE,RO]{B205} %page number on all pages
\fancyfoot[CE,CO]{}

% change first page of all chapters header and footer to fancy style
\makeatletter
\let\ps@plain\ps@fancy
\makeatother

% Do not stretch the content of a page. Instead,
% insert white space at the bottom of the page
\raggedbottom

% Enable arithmetics with length. Useful when typesetting the layout.
\usepackage{calc}
%---------------------------------------------------

%-------BIBLIOGRAPHY--------------------------------
%setting references (using numbers) and supporting i.a. Chicargo-style:
\usepackage{etex}
\usepackage{etoolbox}
\usepackage{keyval}
\usepackage{ifthen}
\usepackage{url}
\usepackage{csquotes}
\usepackage[backend=biber, url=true, doi=true, style=numeric, sorting=none]{biblatex}
\addbibresource{setup/bibliography.bib}
%---------------------------------------------------

%-------MISC----------------------------------------
%%% Enables the use FiXme refferences. Syntax: \fxnote{...} %%%
\usepackage[footnote, draft, english, silent, nomargin]{fixme}
%With "final" instead of "draft" an error will ocure for every FiXme under compilation.

%%% allows use of lorem ipsum (generate i.e. pagagraph 1 to 5 with \lipsum[1-5]) %%%
\usepackage{lipsum}

%%% Enables figures with text wrapped tightly around it %%%
\usepackage{wrapfig}

%%% Section debth included in table of contents (1 = down to sections) %%%
\setcounter{tocdepth}{1}

%%% Section debth for numbers (1 = down to sections) %%%
\setcounter{secnumdepth}{1}

\usepackage{tocloft}
\setlength{\cftbeforetoctitleskip}{0 cm}
\renewcommand{\cftpartpresnum}{Part~}
\let\cftoldpartfont\cftpartfont
\renewcommand{\cftpartfont}{\cftoldpartfont\cftpartpresnum}
%---------------------------------------------------

%-------HYPERLINKS----------------------------------
% Enable hyperlinks and insert info into the pdf
% file. Hypperref should be loaded as one of the 
% last packages
\usepackage{nameref}
\usepackage{hyperref}
\hypersetup{%
	%pdfpagelabels=true,%
	plainpages=false,%
	pdfauthor={Author(s)},%
	pdftitle={Title},%
	pdfsubject={Subject},%
	bookmarksnumbered=true,%
	colorlinks,%
	citecolor=aaublue,%
	filecolor=aaublue,%
	linkcolor=aaublue,% you should probably change this to black before printing
	urlcolor=aaublue,%
	pdfstartview=FitH%
}
%---------------------------------------------------

% remove all indentations
\setlength\parindent{0pt}
\parskip 5mm
\usepackage{verbatim}

\definecolor{Gra}{RGB}{230,230,230}

%creates a nice-looking C#-text
\newcommand{\CC}{C\nolinebreak\hspace{-.05em}\raisebox{.3ex}{\scriptsize\text \#} }

%enables multi column lists
\usepackage{multicol}

%enables code-examples
\usepackage{listings}

\definecolor{coolblue}{RGB}{32,95,128}
\definecolor{mygreen}{rgb}{0,0.6,0}
\definecolor{mygray}{rgb}{0.5,0.5,0.5}
\definecolor{mymauve}{rgb}{0.58,0,0.82}
\usepackage{textcomp}
\definecolor{listinggray}{gray}{0.9}
\definecolor{lbcolor}{rgb}{0.9,0.9,0.9}

\lstset{
backgroundcolor=\color{lbcolor},
	tabsize=4,
	rulecolor=,
	language=C,
        basicstyle=\scriptsize,
        upquote=true,
        aboveskip={1.5\baselineskip},
        columns=fixed,
        showstringspaces=false,
        extendedchars=true,
        breaklines=true,
        prebreak = \raisebox{0ex}[0ex][0ex]{\ensuremath{\hookleftarrow}},
        frame=single,
        showtabs=false,
        numbers=left,
        captionpos=b,
        numbersep=5pt,
        numberstyle=\tiny\color{mygray},
        showspaces=false,
        showstringspaces=false,
        identifierstyle=\ttfamily,
        keywordstyle=\color[rgb]{0,0,1},
        commentstyle=\color[rgb]{0.133,0.545,0.133},
        stringstyle=\color[rgb]{0.627,0.126,0.941},
}

\usepackage{enumitem}
%\usepackage[citestyle=authoryear,natbib=true]{biblatex}

% Figures - TIKZ
\usepackage{tikz}
\usepackage[americanresistors,americaninductors,americancurrents, americanvoltages]{circuitikz}

% Wall of text logo
\newcommand{\walloftextalert}[0]{\includegraphics[width=\textwidth]{walloftext.png}}

\usepackage{pdfpages}
\usepackage{lastpage}
\usepackage{epstopdf}

\setlength{\headheight}{21pt}

\hfuzz=\maxdimen
\tolerance = 10000
\hbadness  = 10000

\usepackage{siunitx}
\graphicspath{{./figures/}}
%Vectors
\renewcommand{\vec}[1]{\boldsymbol{\mathbf{#1}}}
\begin{document}
\renewcommand\chaptername{KAPITEL}
\renewcommand\contentsname{Indhold}
\renewcommand\figurename{Figur}
\renewcommand\tablename{Tabel}

\section*{Review of Paper on\\
Control of Distributed Heat Exchanger Systems width
Model Based Optimal References\\
\small Wednesday, 30th of November 2016}
\subsection{Overall Assessment}
Overall the paper is nice with an interesting subject. The pictures are off a good quality which supplements the paper well. 

\subsection{General Comments}

\noindent Variable index that is short for a word/meaning must be written in mathrm mode, to make the index regular text instead of italic \\
 
\noindent In the itemized labels there is a correlation between content and the letters used, H, T and L. \\
However we would like it to be clear to the reader, what it means. \\
Examples: The hydraulic model, H, .. (then you know the assumptions are related to this) \\
Or: make a \textbf{Hydraulic model assumptions} before the list of assumptions. \\
Or: The hydraulic model, shortened H, .. \\

\noindent It is often unclear what the variables represent. Example: figure 2, $q_0$, $q_1$, $q_2$.  It is written after eq (2). It should be mention the first time it is used and not later.\\
Suggestion: make use of indent, as it would be nice for the reader to have paragraphs, as it makes it more structured to read. \\
 
\noindent Vector notation must be implemented. I.e. bold ink for vectors and matrices.\\

\noindent All capital letters in table captions. It should simply be written as figure text is.\\

\noindent Minima or minimum - examples: “Finding a global minima in [...]” “[...] guaranteed to find the global minima.”\\
Suggestion: We suggest “minimum” instead of “minima” under the assumption that there is only one global minimum. \\

\noindent We will suggest to make the tables prettier, because right now they look very messy.

\subsection{Specific Comments}

\noindent Section I - 1st column - 1st paragraph - “[...] buildings represents big part of [...].”
“[...] buildings represents a big part of [...]”\\


\noindent Section I - 2nd column - 1st paragraph - “[...] saving energy in typical Danish district heat networks.”
Suggestion: “[...] saving energy in typical Danish district heating networks.”\\


\noindent Section I - 2nd column - 1st paragraph - “[...] heat-energy flows $Q_a,1$ and $Q_a,2$, into the rooms that equals the heat-energy flows $Q_r,1$ and $Q_r,2$ out of [...]”
Note: It is not clear what the two subscripts, a and r, stands for. This should also be clear when looking at Fig. 1.\\


\noindent Section I - 2nd column - 1st paragraph - “Since this kind of systems mostly operate in steady state, the cost of the transient response of the system is small compared to the cost of steady state operation.”
Thoughts: This sentence could be more clear. If understood correctly, the two things compared are the cost of steady state and the cost of the behaviour from one steady state reference to another. This could be written more explicitly. 
Suggestion: These kinds of systems mostly operates in steady state, as a thermostat is changed rarely. For this reason the cost related to the transient response of the system, i.e. when changing the thermostat, is small compared to the cost of steady state operation, when observing the system for a longer time period.\\


\noindent Section I - 2nd column - 1st paragraph - “Notation: In the following, [...]”
Thoughts: We think this information is redundant considering the mathematical level and the presumed audience. The space could be used for other more pressing explanations.\\


\noindent Section II - 1st column - 1st paragraph - To make a model of the system the modelling is split into three parts: a model of the hydralic part of the system, a model of the thermal part and finally a combined model to yield a complete model of the system.
Suggestion: 5 model words in one sentence - could be refracted. “hydralic” should be “hydraulic”.\\


\noindent Section II - 1st column - 1st paragraph - H3 - “The length of the pipes is sufficiently long, meaning that pressure drops due to form resistance, such as pipe bends, elbows and pipe fitting, can be lumped into the pipe model.”
Thoughts: We see what has been done from the figures, however the reasoning behind the assumptions is unclear. We cannot see the direct relation between what you write in H3 and what you actually want to do and communicate to the reader. Furthermore, why can it be assumed, what effect there is when the pressure drops? What does the pipe length mean in this relation and the effect? It is to unclear.\\


\noindent Section II - Figure 2 and 3 must be 90 degrees counter clockwise to be in same orientation as figure 1, as it is otherwise counter intuitive. Especially with figure 2.\\


\noindent Section II - 1st column - 1st paragraph - “[...] the pipe section $[h^2 mbar / m3, h^2 bar/m^6]$, q is the volumetric flow  [...]”
Thoughts: The units can be confusing, to clarify, the individual units could be specified, such that, when used together it would seem more coherent.
Suggestion: This is taken from another paper as an example: “Lastly, the units used to express the flow $[m3/s]$ and pressure [Pa] are converted to $[m3/h]$ and [Bar], respectively.”
Note: In one unit millibar is used, mbar, and in the other bar is used, we think it should be bar in both.\\


\noindent Section II- 1st column - 1st paragraph: [...] of the controllable valve [%].
Suggestion: As percentage is not a unit, maybe it would be better to write: [...] is the opening degree, in percentages, of the controllable valve. \\


\noindent Figure 3: Suggestion: Make the variables smaller to be able to place them more accurately - it appears scrambled and confusing as it is now.\\


\noindent Section II - 1st column: spelling: thru = through
Grammar note: Through and thru are different spellings of the same word. Thru is the less preferred form, however, and it might be considered out of place outside the most informal contexts. If you’re writing for school or for a job application, for instance, through is definitely the safer choice. $[http://grammarist.com/spelling/through-thru/]$\\


\noindent Section II - 1st column- 1st paragraph: OBS: “Time delays are modelled by The time delays in the transportation of the hot water depend on the water flows in the pipes. In this project the time delays is modelled as shown in (11) [3].”\\


\noindent Section III - 2nd column - 1st paragraph - “[...] steady state $[DKK/s]$, N is the [...]”
Thoughts: $DKK/s$ is presented as a standard unit. We do not think this is technically correct.\\


\noindent Section V - 2nd column - 1st paragraph - Itemize						
Thoughts: This fits better as figure text. However this will make the figure text too long. Consider if it can be implemented as figure text somehow. \\


\noindent References: “Fjernvarmeforsyning af lavenergiområder,”
Suggestion: There is a comma at the end of each title in the references. Maybe this could be removed. \\


\end{document}